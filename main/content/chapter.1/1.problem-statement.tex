\usection{Planteamiento del Problema}

La visión espacial, o visualización, es definida por \citeauthor{piaget1971} \citeyear{piaget1971} como la capacidad de las personas para generar representaciones mentales del espacio. Estas actúan como guías que permiten entender y trabajar con el espacio, ya sea al dibujar un plano o al resolver un problema de geometría. De manera similar \citeauthor{sanjuan2016vision} \citeyear{sanjuan2016vision} define la visión espacial como la habilidad de operar mentalmente con imágenes visuales o espaciales.

% Utilidad e importancia
De acuerdo con \citeauthor{wai2009spatial} \citeyear{wai2009spatial} la importancia de la visión espacial reside en su capacidad para mejorar la comprensión y manipulación del espacio. Esto resulta vital en campos como la arquitectura, el diseño, la ingeniería y las artes visuales. Además, esta habilidad está estrechamente relacionada con el éxito en áreas de ciencias, tecnología, ingeniería y matemáticas (CTIM), como lo demuestran los estudios revisados por \citeauthor{wai2009spatial} \citeyear{wai2009spatial}, que indican que las habilidades espaciales predicen el rendimiento académico en estas disciplinas. Por lo tanto, desarrollar y mejorar la visión espacial no solo beneficia el rendimiento profesional, sino también el desarrollo cognitivo y la resolución de problemas en general.

% Desafios en la enseñanza de conceptos espaciales
% Como se enseña tradicionalmente
% Qué pasa? (Evidencias)
% Por qué pasa? (Causas)
El estudio de \citeauthor{ramos2025ensenanza} \citeyear{ramos2025ensenanza} identifica varios desafíos en la enseñanza de conceptos espaciales en la educación básica. Los resultados muestran una dificultad entre los estudiantes para identificar objetos en el aula y se observa que las actividades de enseñanza no siempre se perciben como claras o contextualmente aplicables. El estudio también destaca una inconsistencia en la frecuencia de las actividades relacionadas con el espacio y un uso irregular de recursos didácticos como mapas y modelos tridimensionales. Se identifica una subutilización de aplicaciones tecnológicas en la enseñanza de estos conceptos. Estos hallazgos indican áreas de mejora en los enfoques educativos actuales para la enseñanza de conceptos espaciales.

% Qué pasa si no se hace nada? (Tendencias)
La literatura, como evidencian \citeauthor{gunderson2012relation} \citeyear{gunderson2012relation} y \citeauthor{hawes2020explains} \citeyear{hawes2020explains}, establece una clara relación entre las habilidades espaciales, el desarrollo cognitivo general y la capacidad para abordar problemas complejos. Se destaca también que el subdesarrollo de estas habilidades podría comprometer significativamente la capacidad de los estudiantes para adaptarse a entornos cambiantes y complejos, así como su rendimiento académico y desarrollo profesional.

% Qué se puede hacer? (Solucion propuesta)
Ante los desafíos identificados y la importancia de las habilidades espaciales, se propone el desarrollo de una aplicación para la enseñanza de conceptos espaciales en estudiantes de ingeniería basada en realidad virtual. Esta aplicación buscará superar las limitaciones de los métodos tradicionales, ofreciendo una experiencia inmersiva e interactiva. Su diseño se centrará en la intuitividad y facilidad de uso, permitiendo a los estudiantes explorar y manipular conceptos espaciales de manera práctica y atractiva. La realidad virtual ofrece la posibilidad de crear entornos tridimensionales simulados que facilitan la comprensión de conceptos abstractos, permitiendo a los estudiantes experimentar y aplicar sus conocimientos en un contexto virtual realista.

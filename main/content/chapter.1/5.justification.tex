\usection{Justificación}
La presente investigación se motiva en la necesidad continua de mejora en los métodos de enseñanza de la ingeniería, particularmente en el desarrollo de la visión espacial, una habilidad crucial para el éxito profesional en este campo. El desarrollo de un entorno de enseñanza basado en realidad virtual (RV) ofrecerá una solución efectiva para superar las limitaciones de los métodos tradicionales.

\begin{itemize}
  \item La visión espacial es un pilar fundamental para la comprensión y aplicación de conceptos complejos dentro del campo de la ingeniería, abarcando el diseño, la construcción y el análisis de estructuras y sistemas. Sin embargo, los métodos de enseñanza tradicionales, que a menudo se basan en representaciones bidimensionales y modelos físicos estáticos, pueden limitar el desarrollo de una comprensión profunda de los conceptos espaciales por parte de los estudiantes. En consecuencia, se observa una brecha significativa entre las habilidades espaciales que demanda el entorno laboral y las que los estudiantes adquieren a través de la educación convencional.
  \item El proyecto aportaría la creación de entornos de enseñanza inmersivos e interactivos mediante realidad virtual para facilitar la visualización y manipulación tridimensional de objetos y conceptos, permitiendo simular escenarios complejos y abstractos de forma práctica y segura, y promoviendo una enseñanza intuitiva y significativa con retroalimentación inmediata.
  \item Los beneficiarios directos serán los estudiantes de ingeniería, quienes mejorarían significativamente su visión espacial para un mayor rendimiento académico y mejor preparación profesional; mientras que los beneficiarios indirectos incluirían las instituciones educativas, que podrían modernizar su enseñanza y atraer talento, y el sector industrial, que se beneficiaría de ingenieros con habilidades espaciales avanzadas, impulsando la innovación y la competitividad.
  \item Alineación con los Objetivos de Desarrollo Sostenible ODS 4, Educación de Calidad (Meta 4.4): El proyecto contribuiría a garantizar una educación inclusiva y equitativa de calidad, y a promover oportunidades de enseñanza permanente para todos.
  \item El proyecto sentaría las bases para futuras investigaciones y desarrollos en educación de ingeniería y afines mediante tecnologías inmersivas como la realidad virtual, permitiendo la creación de un modelo de enseñanza replicable y escalable a otras instituciones y áreas, contribuyendo así a una cultura de innovación y la adopción de tecnologías emergentes para mejorar la calidad educativa.
\end{itemize}

\begin{document}

\pagenumbering{gobble}
\chapter{Marco Metodológico}

\usection{Tipo de Investigación}

\usection{Técnicas e Instrumentos de Recolección de Datos}

\lipsum[8]


\usection{Metodología de Desarrollo Utilizada}


% Guardar el número de página actual
\newcounter{pags}
\setcounter{pags}{\value{page}}

\pagestyle{prelude}
\pagenumbering{roman}
\setcounter{page}{\value{pags}}

% NOTE - Esto sería perfecto si tan solo pusiera el número en la parte de abajo pero bueh se queda así el formato está bello
\tableofcontents
\clearpage
\listoftables
\clearpage
\listoffigures
\clearpage

\newcounter{abstractpage}
\setcounter{abstractpage}{\value{page}}

% Keywords command
\providecommand{\keywords}[1]
{
  \small
  \textbf{\textit{Palabras clave---}} #1
}

\phantomsection
\addcontentsline{toc}{chapter}{Resumen}
\begin{abstract}
  \thispagestyle{plain}
  \setcounter{page}{\value{abstractpage}}
  \lipsum[1]

  \keywords{Lorem ipsum, dolor sit amet, consectetur, adipiscing elit.}
\end{abstract}


\pagestyle{plain}
\pagenumbering{arabic}
\setcounter{page}{1}

\chapter{Introducción}

Veniam ad magna proident occaecat. Ipsum pariatur exercitation sit occaecat quis proident culpa commodo qui veniam nisi. Ex laboris aliquip dolore laborum nisi officia sunt fugiat do.

Ut ullamco reprehenderit elit amet incididunt Lorem sint adipisicing. Occaecat culpa ut mollit veniam laborum laboris amet exercitation magna eu exercitation est. In anim cupidatat non irure qui aute cillum eiusmod deserunt consectetur. Deserunt laboris et dolore incididunt. Ea ut ipsum anim occaecat id.

Cillum aute ex dolore culpa Lorem Lorem dolore id consectetur anim non ex in est. Velit nisi nostrud ex consectetur dolore dolor cillum sint ullamco et. Voluptate nostrud consectetur Lorem velit enim exercitation veniam.

Labore Lorem irure pariatur labore ea veniam ex qui. Cupidatat non culpa magna mollit anim in est culpa aliquip minim consectetur adipisicing anim. Ex aute nostrud adipisicing sunt eiusmod id. Ut laborum ullamco sint proident cillum officia commodo quis dolore irure officia aute voluptate quis.

\chapter{Marco Metodológico}

\usection{Tipo de Investigación}

\usection{Técnicas e Instrumentos de Recolección de Datos}

\lipsum[8]


\usection{Metodología de Desarrollo Utilizada}

\chapter{Marco Metodológico}

\usection{Tipo de Investigación}

\usection{Técnicas e Instrumentos de Recolección de Datos}

\lipsum[8]


\usection{Metodología de Desarrollo Utilizada}

\chapter{Marco Metodológico}

\usection{Tipo de Investigación}

\usection{Técnicas e Instrumentos de Recolección de Datos}

\lipsum[8]


\usection{Metodología de Desarrollo Utilizada}

\chapter{Marco Metodológico}

\usection{Tipo de Investigación}

\usection{Técnicas e Instrumentos de Recolección de Datos}

\lipsum[8]


\usection{Metodología de Desarrollo Utilizada}

\chapter{Marco Metodológico}

\usection{Tipo de Investigación}

\usection{Técnicas e Instrumentos de Recolección de Datos}

\lipsum[8]


\usection{Metodología de Desarrollo Utilizada}


\phantomsection
\bibliography{references}

\chapter{Marco Metodológico}

\usection{Tipo de Investigación}

\usection{Técnicas e Instrumentos de Recolección de Datos}

\lipsum[8]


\usection{Metodología de Desarrollo Utilizada}


\end{document}

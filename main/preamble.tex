\usepackage[spanish]{babel}
\usepackage[a4paper, margin=1in]{geometry}
\usepackage{graphicx}
\graphicspath{{images/}}
\usepackage{times}
\usepackage{setspace}
\linespread{1.5}
\usepackage{tabularx}
\usepackage[colorlinks=true, linkcolor=black, citecolor=black, urlcolor=black]{hyperref}
\usepackage{apacite}
\bibliographystyle{apacite}
\usepackage{lipsum} % For placeholder text

% Formato para los capítulos
\usepackage{titlesec}
\titlespacing{\chapter}{0pt}{0pt}{40pt}

% Formato para evitar la separación de palabras 
\tolerance=1
\emergencystretch=\maxdimen
\hyphenpenalty=10000
\hbadness=10000

% Formato de números de páginas
\usepackage{fancyhdr}
\fancyhf{}
\setlength{\headheight}{15pt}

\fancypagestyle{prelude}{
  \fancyhf{}
  \fancyfoot[C]{\thepage}
  \renewcommand{\headrulewidth}{0pt}
  \renewcommand{\footrulewidth}{0pt}
}

\fancypagestyle{plain}{
  \fancyhf{}
  \fancyhead[R]{\thepage}
}

\renewcommand{\headrulewidth}{0pt}
\renewcommand{\thechapter}{\Roman{chapter}}

% Formato de sangrías y listas
\usepackage{enumitem}
\renewcommand{\labelitemi}{$\bullet$}
\setlength{\parindent}{0.4in}
\setlist[itemize,1]{leftmargin=0.55in}
\setlist[itemize]{leftmargin=0.2in}

% Formato de la tabla de contenido
\usepackage{tocloft}
\addto\captionsspanish{
  \renewcommand{\contentsname}{Índice de Contenido}
  \renewcommand{\listfigurename}{Índice de Figuras}
  \renewcommand{\listtablename}{Índice de Tablas}
}
\renewcommand{\cfttoctitlefont}{\hfill\normalsize\bfseries}
\renewcommand{\cftaftertoctitle}{\hfill}
\renewcommand{\cftloftitlefont}{\hfill\normalsize\bfseries}
\renewcommand{\cftafterloftitle}{\hfill}
\renewcommand{\cftlottitlefont}{\hfill\normalsize\bfseries}
\renewcommand{\cftafterlottitle}{\hfill}
\setlength{\cftbeforetoctitleskip}{-20pt}
\setlength{\cftbeforeloftitleskip}{-20pt}
\setlength{\cftbeforelottitleskip}{-20pt}
\setcounter{secnumdepth}{4}
\setcounter{tocdepth}{4}

% TOC Chapter Style
\cftpagenumbersoff{chapter}
\setlength{\cftbeforechapskip}{5pt}

% TOC Sections Style
\setlength{\cftsecindent}{0pt}
\setlength{\cftsubsecindent}{1.5em}
\setlength{\cftsubsubsecindent}{3em}
\setlength{\cftparaindent}{4.5em}

% Formato de los diferentes niveles de secciones
\newcommand{\upper}[1]{\uppercase{#1}}
\usepackage{titlesec}
\titleformat{\chapter}
{\normalfont\fontsize{12}{15}\bfseries\centering}{}{0.3em}{\upper}[\thispagestyle{empty}]
\titleformat{\section}
{\normalfont\fontsize{12}{15}\bfseries}{\thesection}{1em}{}
\titleformat{\subsection}
{\normalfont\fontsize{12}{15}\bfseries\narrower}{\thesubsection}{1em}{}
\titleformat{\subsubsection}
{\normalfont\fontsize{12}{15}\bfseries\itshape\narrower}{\thesubsubsection}{1em}{}
\titleformat{\paragraph}
{\normalfont\fontsize{12}{15}\itshape\narrower}{\theparagraph}{1em}{}

% Comandos para crear secciones no numeradas con entrada en el índice
\newcommand{\uchapter}[1]{
\addtocounter{chapter}{1}
\phantomsection
\chapter*{Capítulo \thechapter. \break \\ {#1}}
\addcontentsline{toc}{chapter}{Capítulo \thechapter. {#1}}
}
\newcommand{\uextra}[2]{
\addtocounter{chapter}{1}
\phantomsection
\chapter*{{#1} \thechapter. {#2}}
\addcontentsline{toc}{chapter}{{#1} \thechapter. {#2}}
}
\newcommand{\usection}[1]{
  \phantomsection
  \section*{#1}
  \addcontentsline{toc}{section}{#1}
}
\newcommand{\usubsection}[1]{
  \phantomsection
  \subsection*{#1.}
  \addcontentsline{toc}{subsection}{#1}
}
\newcommand{\usubsubsection}[1]{
  \phantomsection
  \subsubsection*{#1.}
  \addcontentsline{toc}{subsubsection}{#1}
}
\newcommand{\uparagraph}[1]{
  \phantomsection
  \paragraph*{#1.}
  \addcontentsline{toc}{paragraph}{#1}
}

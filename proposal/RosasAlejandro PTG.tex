\documentclass[12pt]{article}
\usepackage[spanish]{babel}
\usepackage[a4paper, margin=1in]{geometry}
\usepackage{times}
\usepackage{setspace}
\linespread{1.5}
\usepackage{graphicx}
\graphicspath{{images/}}
\usepackage[nottoc]{tocbibind}
\usepackage[colorlinks=true, linkcolor=black, citecolor=black, urlcolor=black]{hyperref}
\usepackage{tabularx}
\usepackage[table]{xcolor}
\usepackage{amssymb}
\usepackage{placeins}

% Formato de sangrías y listas
\usepackage{enumitem}
\renewcommand{\labelitemi}{$\bullet$}
\setlength{\parindent}{0.3in}
\setlist[itemize,1]{leftmargin=0.45in}
\setlist[itemize]{leftmargin=0.2in}

% Formato de la tabla de contenido
\usepackage{tocloft}
\setlocalecaption{spanish}{contents}{Índice de Contenido}
\renewcommand{\cfttoctitlefont}{\hfill\normalsize\bfseries}
\renewcommand{\cftaftertoctitle}{\hfill}
\renewcommand{\cftsecleader}{\cftdotfill{\cftdotsep}}
\setlength{\cftbeforesecskip}{0pt}
\setlength{\cftbeforesubsecskip}{0pt}
\setlength{\cftbeforesubsubsecskip}{0pt}

% Formato de números de páginas
\usepackage{fancyhdr}
\fancyhf{}
\setlength{\headheight}{30pt}
\fancypagestyle{heading}{
  \fancyhf{}
  \fancyhead[L]{\includegraphics[width=0.4\textwidth]{ucab-logo.png}}
}
\fancypagestyle{plain}{
  \fancyhf{}
  \fancyhead[R]{\thepage}
}
\renewcommand{\headrulewidth}{0pt}

% Formato de los diferentes niveles de secciones
\usepackage{titlesec}
\setcounter{secnumdepth}{5}
\titleformat{\section}{\normalsize\bfseries\centering}{\thesection}{1em}{}
\titleformat{\subsection}{\normalsize\bfseries}{\thesubsection}{1em}{}
\titleformat{\subsubsection}{\normalsize\bfseries\narrower}{\thesubsubsection}{1em}{}
\titleformat{\paragraph}{\normalsize\bfseries\itshape\narrower}{\theparagraph}{1em}{}
\titleformat{\subparagraph}{\normalsize\itshape}{\thesubparagraph}{1em}{}

% Formato de la bibliografía
\usepackage{apacite}
\bibliographystyle{apacite}

% Formato para evitar la separación de palabras 
\tolerance=1
\emergencystretch=\maxdimen
\hyphenpenalty=10000
\hbadness=10000

% Comandos para crear secciones no numeradas con entrada en el índice
\newcommand{\usection}[1]{
  \phantomsection
  \section*{#1}
  \addcontentsline{toc}{section}{#1}
}
\newcommand{\usubsection}[1]{
  \phantomsection
  \subsection*{#1}
  \addcontentsline{toc}{subsection}{#1}
}
\newcommand{\usubsubsection}[1]{
  \phantomsection
  \subsubsection*{#1.}
  % \addcontentsline{toc}{subsubsection}{#1}
}
\newcommand{\uparagraph}[1]{
  \phantomsection
  \paragraph*{#1.}
  \addcontentsline{toc}{paragraph}{#1}
}
\newcommand{\usubparagraph}[1]{
  \phantomsection
  \subparagraph*{#1.}
  \addcontentsline{toc}{subparagraph}{#1}
}

\newcommand\tab[1][1cm]{\hspace*{#1}}

\begin{document}

\pagestyle{heading}
\pagenumbering{gobble}
\newcommand{\academicTutor}{
  Lárez Mata, Jesús José
}

\newcommand{\membrete}{
  \normalsize\selectfont
  Universidad Católica Andrés Bello \par
  Facultad de Ingeniería \par
  Escuela de Ingeniería Informática \par
}

\newcommand{\tutor}[1]{
  \normalsize\selectfont
  \centering
  Tutor Académico: #1
  \vspace{1in}
}

% \newcommand{\titulo}{Realidad Virtual como Herramienta para la Representación y Enseñanza de Conceptos Espaciales en Ingeniería}
% Aplicación para la Enseñanza-Aprendizaje de Conceptos Espaciales en Estudiantes de Ingeniería basada en Realidad Virtual
\newcommand{\titulo}{Aplicación para Enseñanza de Conceptos Espaciales en Estudiantes de Ingeniería basada en Realidad Virtual}

\newcommand{\estudiante}{Rosas Ordaz, Alejandro José}

\title{
  \begin{figure}[h]
    \centering
    \includegraphics[width=0.6\textwidth]{ucab-logo.png}
  \end{figure}
  \membrete
  \vspace*{\fill}
  \begin{center}
    \fontsize{14}{20}\selectfont
    \titulo
  \end{center}
  \vspace*{\fill}
}
\author{\normalsize\selectfont Realizado por: \estudiante}
\date{
  \tutor{Lárez Mata, Jesús José}
  \begin{figure}[b]
    \centering
    Marzo de 2025
  \end{figure}
}

\maketitle
\clearpage

\begin{table}[h]
  \begin{tabularx}{\textwidth}{X X}
     & Ciudad Guayana, 7 de marzo de 2025 \\
  \end{tabularx}
\end{table}

\begin{table}[h]
  \onehalfspacing
  \begin{tabular}{@{}l@{}}
    Señores                           \\
    Consejo de Facultad de Ingeniería \\
    Facultad de Ingeniería            \\
    Universidad Católica Andrés Bello \\
    Presente. -                       \\
  \end{tabular} \\
\end{table}

Por medio de la presente hago constar que estoy dispuesto a supervisar, en calidad de Tutor Académico el Trabajo Experimental de Grado (TEG) titulado: “\titulo”, que será desarrollado por el estudiante:

\begin{itemize}
  \item \estudiante, C.I.N. Insertar
\end{itemize}

Para lo cual solicito la aprobación de este Consejo de Escuela. Así mismo hago constar que he leído el extracto con la descripción de las funciones del Tutor y estoy conforme con la responsabilidad que me corresponde asumir. \\

Atentamente,
\begin{table}[h]
  \onehalfspacing
  \begin{tabularx}{\textwidth}{>{\hsize=.3\hsize}X X}
             & \textbf{Tutor Académico} \\
    \hline
    Nombre   & \academicTutor           \\
    \hline
    C.I.N.   & Insertar                 \\
    \hline
    E-mail   & Insertar                 \\
    \hline
    Teléfono & Insertar                 \\
    \hline
    Fecha    & Insertar                 \\
    \hline
  \end{tabularx}
\end{table}\\

\hfill
Firma: \underline{
  \hspace{5cm}
}

\clearpage
\section*{Planilla Resumen de Datos de la Propuesta – TEG}

\subsection*{Tema Propuesto:}

\begin{figure}[h]
  \includegraphics[width=\textwidth]{nombre-04-24.png}
\end{figure}

\subsection*{Datos del Estudiante:}
\begin{table}[h]
  \doublespacing
  \begin{tabularx}{\textwidth}{X c c c}
    \hline
    \textbf{Apellidos, Nombres} & \textbf{C.I.N.}    & \textbf{Teléfono} & \textbf{e-mail}  \\
    \hline
    \small{\estudiante}         & \small{28.031.941} & \small{Insertar}  & \small{Insertar} \\
    \hline
  \end{tabularx}
\end{table}

\subsection*{Datos del Tutor Académico:}

\begin{table}[h]
  \onehalfspacing
  \begin{tabularx}{\textwidth}{>{\hsize=.3\hsize}X X}
    Nombre                       & \academicTutor                                                                       \\
    C.I.N.                       & Insertar                                                                             \\
    Profesión                    & Insertar                                                                             \\
    Años Experiencia Profesional & Insertar                                                                             \\
    Cargo Actual                 & Insertar                                                                             \\
    E-mail                       & Insertar                                                                             \\
    Teléfonos                    & Insertar \tab Oficina: Insertar                                                      \\
    \hline
    Años de Graduado             &
    Insertar \tab[4cm] Tutor TG \hfill Sí \space\fbox{$\checkmark$} \hfill No \space\fbox{\phantom{a}}                  \\
    \hline
    Profesor UCAB                & Sí \space\fbox{$\checkmark$} No \space\fbox{\phantom{a}} \tab[2cm] Escuela: Insertar \\
  \end{tabularx}
\end{table}

\clearpage

\section*{Historial de Revisiones}

\begin{table}[h]
  \begin{tabularx}{\textwidth}{>{\raggedright\arraybackslash\hsize=.5\hsize}X X}
    \textbf{Nombre Estudiante}                        & \estudiante \\
    \textbf{Título del Trabajo Experimental de Grado} & \titulo     \\
  \end{tabularx}
\end{table}

\newcolumntype{E}{>{\centering\arraybackslash\hsize=\hsize}X}
\newcolumntype{T}{>{\raggedright\arraybackslash}m{6cm}}

\begin{table}[h]
  \doublespacing
  \begin{tabularx}{\textwidth}{|E|T|T|}
    \hline
    \textbf{Fecha} & \centering\arraybackslash\textbf{Razón del Rechazo}                                                                                                                                       & \centering\arraybackslash\textbf{Modificación Realizada} \\
    \hline
    26/04/2025     & Se solicitan ajustes en el título de la propuesta, alcance, limitaciones y antecedentes, corrección de uso del título, cambios de forma con respecto a las viñetas y tabla de cronograma. & Ajustes según las indicaciones dadas.                    \\
    \hline
    28/04/2025     & Se solicitan ajustes en el alcance y limitaciones, revisión en la redacción de justificación y cambio de forma en el indice.                                                              & Ajustes según las indicaciones dadas.                    \\
    \hline
  \end{tabularx}
\end{table}

\clearpage


\thispagestyle{empty}
\tableofcontents
\clearpage

\pagestyle{plain}
\pagenumbering{arabic}
\setcounter{page}{1}
% TODO - Interconectar parrafos para mayor fluidez de lectura

\usection{Planteamiento del Problema}
% Qué es la Visión Espacial? (Concepto)
La visión espacial, o visualización, es definida por \citeauthor{piaget1971} \citeyear{piaget1971} como la capacidad de las personas para generar representaciones mentales del espacio. Estas actúan como guías que permiten entender y trabajar con el espacio, ya sea al dibujar un plano o al resolver un problema de geometría. De manera similar \citeauthor{sanjuan2016vision} \citeyear{sanjuan2016vision} define la visión espacial como la habilidad de operar mentalmente con imágenes visuales o espaciales.

% Utilidad e importancia
De acuerdo con \citeauthor{wai2009spatial} \citeyear{wai2009spatial} la importancia de la visión espacial reside en su capacidad para mejorar la comprensión y manipulación del espacio. Esto resulta vital en campos como la arquitectura, el diseño, la ingeniería y las artes visuales. Además, esta habilidad está estrechamente relacionada con el éxito en áreas de ciencias, tecnología, ingeniería y matemáticas (CTIM), como lo demuestran los estudios revisados por \citeauthor{wai2009spatial} \citeyear{wai2009spatial}, que indican que las habilidades espaciales predicen el rendimiento académico en estas disciplinas. Por lo tanto, desarrollar y mejorar la visión espacial no solo beneficia el rendimiento profesional, sino también el desarrollo cognitivo y la resolución de problemas en general.

% Desafios en la enseñanza de conceptos espaciales
% Como se enseña tradicionalmente
% Qué pasa? (Evidencias)
% Por qué pasa? (Causas)
El estudio de \citeauthor{ramos2025ensenanza} \citeyear{ramos2025ensenanza} identifica varios desafíos en la enseñanza de conceptos espaciales en la educación básica. Los resultados muestran una dificultad entre los estudiantes para identificar objetos en el aula y se observa que las actividades de enseñanza no siempre se perciben como claras o contextualmente aplicables. El estudio también destaca una inconsistencia en la frecuencia de las actividades relacionadas con el espacio y un uso irregular de recursos didácticos como mapas y modelos tridimensionales. Se identifica una subutilización de aplicaciones tecnológicas en la enseñanza de estos conceptos. Estos hallazgos indican áreas de mejora en los enfoques educativos actuales para la enseñanza de conceptos espaciales.

% Qué pasa si no se hace nada? (Tendencias)
La literatura, como evidencian \citeauthor{gunderson2012relation} \citeyear{gunderson2012relation} y \citeauthor{hawes2020explains} \citeyear{hawes2020explains}, establece una clara relación entre las habilidades espaciales, el desarrollo cognitivo general y la capacidad para abordar problemas complejos. Se destaca también que el subdesarrollo de estas habilidades podría comprometer significativamente la capacidad de los estudiantes para adaptarse a entornos cambiantes y complejos, así como su rendimiento académico y desarrollo profesional.

% Qué se puede hacer? (Solucion propuesta)
Ante los desafíos identificados y la importancia de las habilidades espaciales, se propone el desarrollo de una aplicación para la enseñanza de conceptos espaciales en estudiantes de ingeniería basada en realidad virtual. Esta aplicación buscará superar las limitaciones de los métodos tradicionales, ofreciendo una experiencia inmersiva e interactiva. Su diseño se centrará en la intuitividad y facilidad de uso, permitiendo a los estudiantes explorar y manipular conceptos espaciales de manera práctica y atractiva. La realidad virtual ofrece la posibilidad de crear entornos tridimensionales simulados que facilitan la comprensión de conceptos abstractos, permitiendo a los estudiantes experimentar y aplicar sus conocimientos en un contexto virtual realista.

\clearpage

\usection{Objetivos}

\usubsection{Objetivo General}
Desarrollar una aplicación para enseñanza de conceptos espaciales en estudiantes de ingeniería basada en realidad virtual.


\usubsection{Objetivos Específicos}

\newcommand{\oeOne}{Analizar la situacion que presentan los estudiantes de ingenieria al abordar conceptos espaciales, identificando las necesidades y requerimientos para el desarrollo de la aplicación.}
\newcommand{\oeTwo}{Diseñar una aplicación para enseñanza de conceptos espaciales a estudiantes de ingeniería basada en realidad virtual en función del análisis realizado.}
\newcommand{\oeThree}{Implementar la aplicación para enseñanza de conceptos espaciales a estudiantes de ingeniería basada en realidad virtual según el diseño realizado.}
\newcommand{\oeFour}{Validar la aplicación para enseñanza de conceptos espaciales a estudiantes de ingeniería basada en realidad virtual con respecto al análisis realizado}
\newcommand{\oeFive}{Realizar la documentación de la aplicación para enseñanza de conceptos espaciales a estudiantes de ingeniería basada en realidad virtual realizada.}
\begin{enumerate}
  \item \oeOne
  \item \oeTwo
  \item \oeThree
  \item \oeFour
  \item \oeFive
\end{enumerate}

\clearpage

\usection{Alcance}

\lipsum[1-2]

\clearpage

\usection{Limitaciones}

\lipsum[1-2]

\clearpage

\usection{Justificación}
La presente investigación se motiva en la necesidad continua de mejora en los métodos de enseñanza de la ingeniería, particularmente en el desarrollo de la visión espacial, una habilidad crucial para el éxito profesional en este campo. El desarrollo de un entorno de enseñanza basado en realidad virtual (RV) ofrecerá una solución efectiva para superar las limitaciones de los métodos tradicionales.

\begin{itemize}
  \item La visión espacial es un pilar fundamental para la comprensión y aplicación de conceptos complejos dentro del campo de la ingeniería, abarcando el diseño, la construcción y el análisis de estructuras y sistemas. Sin embargo, los métodos de enseñanza tradicionales, que a menudo se basan en representaciones bidimensionales y modelos físicos estáticos, pueden limitar el desarrollo de una comprensión profunda de los conceptos espaciales por parte de los estudiantes. En consecuencia, se observa una brecha significativa entre las habilidades espaciales que demanda el entorno laboral y las que los estudiantes adquieren a través de la educación convencional.
  \item El proyecto aportaría la creación de entornos de enseñanza inmersivos e interactivos mediante realidad virtual para facilitar la visualización y manipulación tridimensional de objetos y conceptos, permitiendo simular escenarios complejos y abstractos de forma práctica y segura, y promoviendo una enseñanza intuitiva y significativa con retroalimentación inmediata.
  \item Los beneficiarios directos serán los estudiantes de ingeniería, quienes mejorarían significativamente su visión espacial para un mayor rendimiento académico y mejor preparación profesional; mientras que los beneficiarios indirectos incluirían las instituciones educativas, que podrían modernizar su enseñanza y atraer talento, y el sector industrial, que se beneficiaría de ingenieros con habilidades espaciales avanzadas, impulsando la innovación y la competitividad.
  \item Alineación con los Objetivos de Desarrollo Sostenible ODS 4, Educación de Calidad (Meta 4.4): El proyecto contribuiría a garantizar una educación inclusiva y equitativa de calidad, y a promover oportunidades de enseñanza permanente para todos.
  \item El proyecto sentaría las bases para futuras investigaciones y desarrollos en educación de ingeniería y afines mediante tecnologías inmersivas como la realidad virtual, permitiendo la creación de un modelo de enseñanza replicable y escalable a otras instituciones y áreas, contribuyendo así a una cultura de innovación y la adopción de tecnologías emergentes para mejorar la calidad educativa.
\end{itemize}

\clearpage

% Conjunto de investigaciones realizadas sobre el tema en estudio que estén relacionados con la propuesta y cómo pueden sustentar o contribuir con la propuesta. Permiten conocer el estado del arte y los resultados más importantes del tema. Así como los principales conceptos que sustentan el desarrollo del proyecto.
\usection{Marco Referencial}

\usubsection{Antecedentes}
\subsubsection*{Immersive interfaces for engagement and learning [Interfaces inmersivas para la participación y el aprendizaje]}
En este estudio, \citeauthor{dede2009immersive} \citeyear{dede2009immersive} se enfoca en el diseño de experiencias educativas que aprovechan la fluidez digital de los estudiantes para promover el compromiso, la enseñanza y la transferencia de conocimientos a entornos del mundo real, esto mediante el uso de interfaces inmersivas, las cuales generan en el usuario la impresión subjetiva de que está participando en una experiencia realista e integral. Este estudio se distingue por su exploración en cómo la inmersión en un entorno digital puede mejorar la educación al permitir múltiples perspectivas, la enseñanza situada y la transferencia de conocimientos. Además, resalta la importancia de diseñar experiencias de enseñanza inmersivas que combinen factores sensoriales, de acción y simbólicos para aumentar la sensación de presencia del participante en el entorno virtual.

Presentando así, esta investigación los siguientes puntos clave en relación con la enseñanza inmersiva:

% NOTE - Aprendizaje situado
\begin{itemize}
  \item Destaca cómo la capacidad de cambiar la perspectiva en un entorno virtual facilita la comprensión de fenómenos complejos, alternando entre vistas exocéntricas y egocéntricas para combinar las fortalezas de ambas.
  \item Enfatiza el potencial de las interfaces inmersivas para simular comunidades de resolución de problemas auténticas, donde los estudiantes interactúan con otros participantes y agentes virtuales, fomentando la enseñanza situada.
  \item Ejemplifica cómo la inmersión en simulaciones virtuales, como River City, puede mejorar la enseñanza de habilidades de indagación de orden superior, como la formulación de hipótesis y el diseño experimental, en comparación con la instrucción convencional.
  \item Sugiere que la inmersión digital puede ayudar a los estudiantes de bajo rendimiento a desarrollar confianza en sus habilidades académicas al permitirles asumir roles de éxito en un contexto virtual, lo que podría liberar inteligencia y compromiso en muchos estudiantes.
  \item Explora el uso de la realidad aumentada como una forma de inmersión que combina entornos reales y virtuales, y cómo esta tecnología puede mejorar la enseñanza en áreas como matemáticas y alfabetización.
\end{itemize}

Tomando en cuenta estos puntos, se busca aplicar la inmersión en el entorno virtual para mejorar la enseñanza de conceptos matemáticos, específicamente en el ámbito de la geometría analítica. La investigación de \citeauthor{dede2009immersive} proporciona un marco conceptual valioso para el desarrollo de experiencias educativas inmersivas que fomenten la participación activa y el aprendizaje significativo.

\subsubsection*{Desarrollo de habilidades espaciales en estudiantes de ingeniería mediante CAD especializado}
La tesis doctoral de \citeauthor{tristancho2019desarrollo} \citeyear{tristancho2019desarrollo} se centra en el estudio de la habilidad espacial en estudiantes de ingeniería, su importancia en el éxito académico y profesional, y cómo puede desarrollarse a través de la educación y la práctica. La investigación incluye la evaluación de habilidades espaciales en estudiantes de primer curso de ingeniería de la UPC, utilizando pruebas como DAT-SR, PSVT:R y MCT, y la realización de entrevistas para analizar las estrategias de resolución de tareas espaciales.

Con esto es posible identificar que técnicas e implementaciones de CAD pueden ayudar a la enseñanza de conceptos matemáticos, en este caso, el uso de software CAD para la enseñanza de la geometría analítica. Este antecedente muestra las distintas herramientas e implementaciones de valor en un entorno no inmersivo y se busca realizar de cierta manera una evolución del mismo esta vez inmersivo.

\subsubsection*{GeoGebra, una propuesta para innovar el proceso enseñanza-aprendizaje en matemáticas}
El estudio de \citeauthor{garcia2017geogebra} \citeyear{garcia2017geogebra} destaca cómo la introducción de la tecnología en el aula de matemáticas ha generado un debate sobre las ventajas de su uso en el proceso de enseñanza-aprendizaje. Los autores señalan que las TIC pueden ser beneficiosas tanto para el alumno como para el docente, ya que permiten desarrollar el pensamiento matemático y las habilidades digitales necesarias para innovar en el aula. Además, el estudio resalta la importancia de que los docentes adquieran competencias digitales y planifiquen el currículo integrando las TIC de manera efectiva.

Dicho estudio muestra un antecedente sobre como herramientas de graficación no inmersivas pueden ayudar a la enseñanza de conceptos matemáticos, en este caso, el uso de GeoGebra para la enseñanza de la geometría analítica. Este antecedente muestra las distintas herramientas e implementaciones de valor en un entorno no inmersivo y se busca realizar de cierta manera una evolución del mismo esta vez inmersivo.

\subsubsection*{Propuesta Metodológica para la Elaboración de Software Educativo, en Realidad Virtual, bajo los Conceptos Constructivistas}
En esta investigación, \citeauthor{fuenmayor} \citeyear{fuenmayor} analiza cómo la realidad virtual puede incorporarse en el proceso educativo bajo los principios del constructivismo. El estudio propone un marco metodológico para el desarrollo de aplicaciones educativas en entornos virtuales, basado en fundamentos pedagógicos y constructivistas, con el propósito de facilitar la asimilación de ideas complejas y abstractas en estudiantes de edades tempranas.

El desarrollo de la aplicación será basado en dicho marco metodológico, debido a que demuestra estar alineado con el enfoque y objetivos del proyecto, priorizando al usuario como eje central del proceso.

\subsubsection*{Prototipo basado en técnicas de Realidad Virtual semi-inmersiva para la enseñanza de Cálculo Multivariado y Vectorial en estudiantes de Ingeniería}
El estudio de \citeauthor{raigoza2017prototipo} \citeyear{raigoza2017prototipo} presenta el diseño, implementación y validación funcional de un prototipo de software basado en RV semi-inmersiva para la enseñanza del Cálculo Multivariado y Vectorial en estudiantes de ingeniería. El prototipo contiene una serie de escenarios (salones) con ejemplos de conceptos de dicho curso, que son visualizados usando smartphones con sistema operativo Android y Cardboards (gafas estereoscópicas de bajo costo). Los resultados de las pruebas experimentales arrojaron un alto grado de confiabilidad y aceptación por parte de los estudiantes. Dicho prototipo expone los beneficios de utilizar la RV en el aula y que si es posible enseñar de esta manera. Este antecedente es relevante para el desarrollo ya que se busca realizar una nueva visión del prototipo propuesto inmersivo y con nuevas características, apoyándose del proceso y la experiencia obtenida en este trabajo.

% NOTE - Inclusion: Dar una forma diferente de atacar el problema


\usubsection{Bases Teóricas}
Es importante conocer el significado de una serie de conceptos fundamentales para el desarrollo de la propuesta, a manera de clarificar las actividades que hacen posible la construcción de la aplicación.

% TODO - Desarrollar Bases teoricas
\usubsubsection{Visión Espacial}

\usubsubsection{Constructivismo}

\usubsubsection{Inmersión y Entornos Inmersivos}

\usubsubsection{Realidad Virtual}

% Según \cite{handa2012immersive}, la Realidad Virtual (RV) es una tecnología que permite a los usuarios sumergirse en un entorno simulado generado por computadora. La RV se ha convertido en una herramienta poderosa para la visualización de datos y la simulación de entornos. La RV se ha utilizado en una variedad de campos, incluidos la medicina, la educación, la arquitectura y la industria del entretenimiento. La RV se ha convertido en una herramienta poderosa para la visualización de datos y la simulación de entornos. La RV se ha utilizado en una variedad de campos, incluidos la medicina, la educación, la arquitectura y la industria del entretenimiento. La RV se ha convertido en una herramienta poderosa para la visualización de datos y la simulación de entornos. La RV se ha utilizado en una variedad de campos, incluidos la medicina, la educación, la arquitectura y la industria del entretenimiento.

\uparagraph{Realidad Virtual en la Educación}

\uparagraph{Dispositivos de Realidad Virtual}

\usubparagraph{Meta Quest}


\clearpage

\usection{Procedimiento Metodológico}

\lipsum[2]

\usubsection{Justificación de la Metodología}

\lipsum[3]

\clearpage

% Descripción detallada de la planificación del proyecto. Por cada objetivo específico: descripción de tareas a realizar para alcanzarlo, destacando producto y su distribución en el tiempo; aquí se definen actividades y entregables a medida que el proyecto avanza; ver Anexo A2 – Ejemplo Referencial de Cronograma de Actividades
\usection{Cronograma de Trabajo}

% Color Column
\newcommand{\cc}{\cellcolor[HTML]{4bacc6}}
\newcolumntype{L}{>{\raggedright\arraybackslash\hsize=.435\hsize}X}
\newcolumntype{S}{>{\centering\arraybackslash\hsize=.045\hsize}X}

\newcommand{\tableObjective}[2]{
  \vspace{-2em}
  \begin{table}[!ht]
    \begin{tabularx}{\textwidth}{|X|}
      \hline
      \textbf{#1.} #2 \\
      \hline
    \end{tabularx}
  \end{table}
  \FloatBarrier
  \vspace{-2em}
}

\newcommand{\workScheduleHeader}{
  \begin{table}[!ht]
    \onehalfspacing
    \hfill
    \begin{tabularx}{0.63\textwidth}{|X|}
      \hline
      \centering\arraybackslash\textbf{Bloque Semanal (2 Semanas / Bloque)} \\
      \hline
    \end{tabularx}
  \end{table}
  \vspace{-2em}
  \begin{table}[!ht]
    \onehalfspacing
    \begin{tabularx}{\textwidth}{|L|S|S|S|S|S|S|S|S|S|S|}
      \hline
      \centering\arraybackslash\textbf{Actividades por Objetivo} & \textbf{1} & \textbf{2} & \textbf{3} & \textbf{4} & \textbf{5} & \textbf{6} & \textbf{7} & \textbf{8} & \textbf{9} & \textbf{10} \\
      \hline
    \end{tabularx}
  \end{table}
}

\workScheduleHeader
\tableObjective{1}{\oeOne}
\begin{table}[!ht]
  \begin{tabularx}{\textwidth}{|L|S|S|S|S|S|S|S|S|S|S|}
    \hline
    Analizar la situación que presentan los estudiantes de ingeniería al abordar conceptos espaciales para identificar necesidades y requerimientos generales en la aplicación. & \cc & \cc &     &  &  &  &  &  &  & \\
    \hline
    Definir los requerimientos funcionales y no funcionales de la aplicación a desarrollar con respecto al análisis general hecho anteriormente.                                &     &     & \cc &  &  &  &  &  &  & \\
    \hline
  \end{tabularx}
\end{table}
\tableObjective{2}{\oeTwo}
\begin{table}[!ht]
  \begin{tabularx}{\textwidth}{|L|S|S|S|S|S|S|S|S|S|S|}
    \hline
    Diseñar la arquitectura y la interfaz de usuario de la aplicación. &  &  & \cc & \cc &  &  &  &  &  & \\
    \hline
    Diseñar la lógica y funcionalidad de la aplicación.                &  &  & \cc & \cc &  &  &  &  &  & \\
    \hline
    Diseñar protocolo de pruebas.                                      &  &  & \cc & \cc &  &  &  &  &  & \\
    \hline
  \end{tabularx}
\end{table}

\tableObjective{3}{\oeThree}
\begin{table}[!ht]
  \begin{tabularx}{\textwidth}{|L|S|S|S|S|S|S|S|S|S|S|}
    \hline
    Implementar módulos definidos, de acuerdo al diseño realizado. &  &  &  &  & \cc & \cc & \cc & \cc &  & \\
    \hline
    Aplicar protocolo de pruebas diseñado.                         &  &  &  &  & \cc & \cc & \cc & \cc &  & \\
    \hline
    Realizar ajustes de acuerdo a pruebas realizadas.              &  &  &  &  & \cc & \cc & \cc & \cc &  & \\
    \hline
  \end{tabularx}
\end{table}
\clearpage
\workScheduleHeader
\tableObjective{4}{\oeFour}
\begin{table}[!ht]
  \begin{tabularx}{\textwidth}{|L|S|S|S|S|S|S|S|S|S|S|}
    \hline
    Realizar pruebas y validación con usuarios.           &  &  &  &  &  &  &  &  & \cc & \\
    \hline
    Realizar ajustes de acuerdo a validación de usuarios. &  &  &  &  &  &  &  &  & \cc & \\
    \hline
  \end{tabularx}
\end{table}
\tableObjective{5}{\oeFive}
\begin{table}[ht!]
  \begin{tabularx}{\textwidth}{|L|S|S|S|S|S|S|S|S|S|S|}
    \hline
    Realizar documentación técnica.    &  &  &  &  &  &  & \cc & \cc & \cc & \cc \\
    \hline
    Realizar documentación de usuario. &  &  &  &  &  &  &     &     & \cc & \cc \\
    \hline
  \end{tabularx}
\end{table}

\clearpage

\phantomsection
\bibliography{references}

\end{document}

\usection{Alcance}

El presente trabajo se enfocará en el desarrollo de una aplicación basada en realidad virtual para la enseñanza de conceptos espaciales, orientada a fortalecer la visión espacial en estudiantes de ingeniería. La aplicación estará dirigida a estudiantes que cursen asignaturas relacionadas con conceptos matemáticos y geométricos, y se desarrollará utilizando tecnologías de Realidad Virtual Inmersiva, haciendo uso de dispositivos Meta Quest. El desarrollo se llevará a cabo en la Universidad Católica Andrés Bello en un plazo de 20 semanas, teniendo en cuenta lo siguiente:

\begin{itemize}
  \item Se realizará un análisis y levantamiento de requerimientos para identificar las necesidades de los usuarios y definir las funcionalidades a implementar en la aplicación.
  \item Se diseñará la arquitectura de la aplicación, definiendo los componentes y su interacción, así como la interfaz de usuario.
  \item Se implementará la aplicación utilizando el motor gráfico Unity, haciendo uso de un dispositivo de realidad virtual Meta Quest, integrando los modelos 3D y las funcionalidades definidas en la fase de diseño.
  \item Se hará una validación de la aplicación con usuarios reales, recolectando feedback para realizar mejoras y ajustes necesarios.
  \item Se realizará una documentación de la aplicación, incluyendo el manual de usuario y la guía de instalación.
        % NOTE
        % \item El producto final permitirá a los usuarios interactuar con objetos tridimensionales en un entorno virtual inmersivo, facilitando la comprensión de conceptos espaciales, aplicando:
        %       \begin{itemize}
        %         \item Interfaces de usuario intuitivas.
        %         \item Diagramado de ecuaciones escritas por el usuario.
        %         \item Libertad de movimiento por el entorno virtual generado.
        %         \item Herramientas de manipulación de los objetos generados.
        %         \item Parametrización de los objetos.
        %         \item Visualización de puntos de intersección.
        %         \item Uso de deslizadores para la visualización de ecuaciones en distintos estados cuando la variable toma un valor dado.
        %         \item Representación de vectores y magnitudes en el espacio.
        %         \item Representación de campos vectoriales y escalares en el espacio.
        %       \end{itemize}
\end{itemize}

\clearpage

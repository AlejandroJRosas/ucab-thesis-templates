\subsubsection*{Immersive interfaces for engagement and learning [Interfaces inmersivas para la participación y el aprendizaje]}
En este estudio, \citeauthor{dede2009immersive} \citeyear{dede2009immersive} se enfoca en el diseño de experiencias educativas que aprovechan la fluidez digital de los estudiantes para promover el compromiso, la enseñanza y la transferencia de conocimientos a entornos del mundo real, esto mediante el uso de interfaces inmersivas, las cuales generan en el usuario la impresión subjetiva de que está participando en una experiencia realista e integral. Este estudio se distingue por su exploración en cómo la inmersión en un entorno digital puede mejorar la educación al permitir múltiples perspectivas, la enseñanza situada y la transferencia de conocimientos. Además, resalta la importancia de diseñar experiencias de enseñanza inmersivas que combinen factores sensoriales, de acción y simbólicos para aumentar la sensación de presencia del participante en el entorno virtual.

Presentando así, esta investigación los siguientes puntos clave en relación con la enseñanza inmersiva:

% NOTE - Aprendizaje situado
\begin{itemize}
  \item Destaca cómo la capacidad de cambiar la perspectiva en un entorno virtual facilita la comprensión de fenómenos complejos, alternando entre vistas exocéntricas y egocéntricas para combinar las fortalezas de ambas.
  \item Enfatiza el potencial de las interfaces inmersivas para simular comunidades de resolución de problemas auténticas, donde los estudiantes interactúan con otros participantes y agentes virtuales, fomentando la enseñanza situada.
  \item Ejemplifica cómo la inmersión en simulaciones virtuales, como River City, puede mejorar la enseñanza de habilidades de indagación de orden superior, como la formulación de hipótesis y el diseño experimental, en comparación con la instrucción convencional.
  \item Sugiere que la inmersión digital puede ayudar a los estudiantes de bajo rendimiento a desarrollar confianza en sus habilidades académicas al permitirles asumir roles de éxito en un contexto virtual, lo que podría liberar inteligencia y compromiso en muchos estudiantes.
  \item Explora el uso de la realidad aumentada como una forma de inmersión que combina entornos reales y virtuales, y cómo esta tecnología puede mejorar la enseñanza en áreas como matemáticas y alfabetización.
\end{itemize}

Tomando en cuenta estos puntos, se busca aplicar la inmersión en el entorno virtual para mejorar la enseñanza de conceptos matemáticos, específicamente en el ámbito de la geometría analítica. La investigación de \citeauthor{dede2009immersive} proporciona un marco conceptual valioso para el desarrollo de experiencias educativas inmersivas que fomenten la participación activa y el aprendizaje significativo.

\subsubsection*{Desarrollo de habilidades espaciales en estudiantes de ingeniería mediante CAD especializado}
La tesis doctoral de \citeauthor{tristancho2019desarrollo} \citeyear{tristancho2019desarrollo} se centra en el estudio de la habilidad espacial en estudiantes de ingeniería, su importancia en el éxito académico y profesional, y cómo puede desarrollarse a través de la educación y la práctica. La investigación incluye la evaluación de habilidades espaciales en estudiantes de primer curso de ingeniería de la UPC, utilizando pruebas como DAT-SR, PSVT:R y MCT, y la realización de entrevistas para analizar las estrategias de resolución de tareas espaciales.

Con esto es posible identificar que técnicas e implementaciones de CAD pueden ayudar a la enseñanza de conceptos matemáticos, en este caso, el uso de software CAD para la enseñanza de la geometría analítica. Este antecedente muestra las distintas herramientas e implementaciones de valor en un entorno no inmersivo y se busca realizar de cierta manera una evolución del mismo esta vez inmersivo.

\subsubsection*{GeoGebra, una propuesta para innovar el proceso enseñanza-aprendizaje en matemáticas}
El estudio de \citeauthor{garcia2017geogebra} \citeyear{garcia2017geogebra} destaca cómo la introducción de la tecnología en el aula de matemáticas ha generado un debate sobre las ventajas de su uso en el proceso de enseñanza-aprendizaje. Los autores señalan que las TIC pueden ser beneficiosas tanto para el alumno como para el docente, ya que permiten desarrollar el pensamiento matemático y las habilidades digitales necesarias para innovar en el aula. Además, el estudio resalta la importancia de que los docentes adquieran competencias digitales y planifiquen el currículo integrando las TIC de manera efectiva.

Dicho estudio muestra un antecedente sobre como herramientas de graficación no inmersivas pueden ayudar a la enseñanza de conceptos matemáticos, en este caso, el uso de GeoGebra para la enseñanza de la geometría analítica. Este antecedente muestra las distintas herramientas e implementaciones de valor en un entorno no inmersivo y se busca realizar de cierta manera una evolución del mismo esta vez inmersivo.

\subsubsection*{Propuesta Metodológica para la Elaboración de Software Educativo, en Realidad Virtual, bajo los Conceptos Constructivistas}
En esta investigación, \citeauthor{fuenmayor} \citeyear{fuenmayor} analiza cómo la realidad virtual puede incorporarse en el proceso educativo bajo los principios del constructivismo. El estudio propone un marco metodológico para el desarrollo de aplicaciones educativas en entornos virtuales, basado en fundamentos pedagógicos y constructivistas, con el propósito de facilitar la asimilación de ideas complejas y abstractas en estudiantes de edades tempranas.

El desarrollo de la aplicación será basado en dicho marco metodológico, debido a que demuestra estar alineado con el enfoque y objetivos del proyecto, priorizando al usuario como eje central del proceso.

\subsubsection*{Prototipo basado en técnicas de Realidad Virtual semi-inmersiva para la enseñanza de Cálculo Multivariado y Vectorial en estudiantes de Ingeniería}
El estudio de \citeauthor{raigoza2017prototipo} \citeyear{raigoza2017prototipo} presenta el diseño, implementación y validación funcional de un prototipo de software basado en RV semi-inmersiva para la enseñanza del Cálculo Multivariado y Vectorial en estudiantes de ingeniería. El prototipo contiene una serie de escenarios (salones) con ejemplos de conceptos de dicho curso, que son visualizados usando smartphones con sistema operativo Android y Cardboards (gafas estereoscópicas de bajo costo). Los resultados de las pruebas experimentales arrojaron un alto grado de confiabilidad y aceptación por parte de los estudiantes. Dicho prototipo expone los beneficios de utilizar la RV en el aula y que si es posible enseñar de esta manera. Este antecedente es relevante para el desarrollo ya que se busca realizar una nueva visión del prototipo propuesto inmersivo y con nuevas características, apoyándose del proceso y la experiencia obtenida en este trabajo.

% NOTE - Inclusion: Dar una forma diferente de atacar el problema

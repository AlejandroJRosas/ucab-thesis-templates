\usection{Procedimiento Metodológico}
Para el desarrollo de la aplicación de Realidad Virtual enfocada en la enseñanza de conceptos espaciales en ingeniería, nos basaremos en la metodología de desarrollo ágil, específicamente la metodología Extreme Us propuesta por \citeauthor{fuenmayor} \citeyear{fuenmayor}, combinada con principios de Diseño Centrado en el Usuario (DCU). Este enfoque se alinea con la naturaleza interactiva, evolutiva y educativa de la Realidad Virtual, permitiendo adaptabilidad y mejoras continuas basadas en la retroalimentación del usuario y los objetivos pedagógicos.

\usubsection{Justificación de la Metodología}
La elección de Extreme Us se fundamenta en su flexibilidad y capacidad para adaptarse a los cambios, lo cual es crucial en proyectos de Realidad Virtual donde la iteración, la experimentación y la validación educativa son clave. Según \citeauthor{fuenmayor} \citeyear{fuenmayor}, Extreme Us promueve la comunicación constante, la retroalimentación temprana y la simplicidad en el diseño, lo que facilita la creación de un entorno de enseñanza virtual intuitivo, efectivo y alineado con los principios pedagógicos y constructivistas.

\usubsection{Fases del Modelo Extreme Us}
\begin{enumerate}
  \item Planificación (Planning Game): Se definen los objetivos de enseñanza, el contenido y la estructura general de la aplicación, involucrando a educadores y estudiantes en el proceso. Se priorizan las funcionalidades y se establecen las historias de usuario.
  \item Diseño: Se crean prototipos y bocetos del entorno virtual, considerando la usabilidad y la experiencia del usuario. Se diseñan las interacciones y se planifica la evaluación continua del diseño.
  \item Codificación: Se desarrolla el entorno virtual en iteraciones cortas, implementando las funcionalidades pedagógicas priorizadas en la fase de planificación. Se realizan revisiones técnicas y pedagógicas al final de cada iteración para asegurar la calidad del desarrollo y la alineación con los objetivos de enseñanza.
  \item Pruebas y Retroalimentación: Se realizan pruebas con usuarios reales para identificar áreas de mejora y se recopila retroalimentación para iterar y refinar el diseño y la funcionalidad de la aplicación.
  \item Evaluación: Se evalúa la efectividad de la aplicación en la enseñanza de conceptos espaciales, utilizando métricas cuantitativas y cualitativas para medir el impacto en los estudiantes.
\end{enumerate}

Este enfoque metodológico permite una adaptación continua y asegura que la aplicación de Realidad Virtual sea una solución efectiva y centrada en el usuario para la enseñanza de conceptos espaciales en ingeniería.

\clearpage

% Descripción detallada de la planificación del proyecto. Por cada objetivo específico: descripción de tareas a realizar para alcanzarlo, destacando producto y su distribución en el tiempo; aquí se definen actividades y entregables a medida que el proyecto avanza; ver Anexo A2 – Ejemplo Referencial de Cronograma de Actividades
\usection{Cronograma de Trabajo}

% Color Column
\newcommand{\cc}{\cellcolor[HTML]{4bacc6}}
\newcolumntype{L}{>{\raggedright\arraybackslash\hsize=.435\hsize}X}
\newcolumntype{S}{>{\centering\arraybackslash\hsize=.045\hsize}X}

\newcommand{\tableObjective}[2]{
  \vspace{-2em}
  \begin{table}[!ht]
    \begin{tabularx}{\textwidth}{|X|}
      \hline
      \textbf{#1.} #2 \\
      \hline
    \end{tabularx}
  \end{table}
  \FloatBarrier
  \vspace{-2em}
}

\newcommand{\workScheduleHeader}{
  \begin{table}[!ht]
    \onehalfspacing
    \hfill
    \begin{tabularx}{0.63\textwidth}{|X|}
      \hline
      \centering\arraybackslash\textbf{Bloque Semanal (2 Semanas / Bloque)} \\
      \hline
    \end{tabularx}
  \end{table}
  \vspace{-2em}
  \begin{table}[!ht]
    \onehalfspacing
    \begin{tabularx}{\textwidth}{|L|S|S|S|S|S|S|S|S|S|S|}
      \hline
      \centering\arraybackslash\textbf{Actividades por Objetivo} & \textbf{1} & \textbf{2} & \textbf{3} & \textbf{4} & \textbf{5} & \textbf{6} & \textbf{7} & \textbf{8} & \textbf{9} & \textbf{10} \\
      \hline
    \end{tabularx}
  \end{table}
}

\workScheduleHeader
\tableObjective{1}{\oeOne}
\begin{table}[!ht]
  \begin{tabularx}{\textwidth}{|L|S|S|S|S|S|S|S|S|S|S|}
    \hline
    Analizar la situación que presentan los estudiantes de ingeniería al abordar conceptos espaciales para identificar necesidades y requerimientos generales en la aplicación. & \cc & \cc &     &  &  &  &  &  &  & \\
    \hline
    Definir los requerimientos funcionales y no funcionales de la aplicación a desarrollar con respecto al análisis general hecho anteriormente.                                &     &     & \cc &  &  &  &  &  &  & \\
    \hline
  \end{tabularx}
\end{table}
\tableObjective{2}{\oeTwo}
\begin{table}[!ht]
  \begin{tabularx}{\textwidth}{|L|S|S|S|S|S|S|S|S|S|S|}
    \hline
    Diseñar la arquitectura y la interfaz de usuario de la aplicación. &  &  & \cc & \cc &  &  &  &  &  & \\
    \hline
    Diseñar la lógica y funcionalidad de la aplicación.                &  &  & \cc & \cc &  &  &  &  &  & \\
    \hline
    Diseñar protocolo de pruebas.                                      &  &  & \cc & \cc &  &  &  &  &  & \\
    \hline
  \end{tabularx}
\end{table}

\tableObjective{3}{\oeThree}
\begin{table}[!ht]
  \begin{tabularx}{\textwidth}{|L|S|S|S|S|S|S|S|S|S|S|}
    \hline
    Implementar módulos definidos, de acuerdo al diseño realizado. &  &  &  &  & \cc & \cc & \cc & \cc &  & \\
    \hline
    Aplicar protocolo de pruebas diseñado.                         &  &  &  &  & \cc & \cc & \cc & \cc &  & \\
    \hline
    Realizar ajustes de acuerdo a pruebas realizadas.              &  &  &  &  & \cc & \cc & \cc & \cc &  & \\
    \hline
  \end{tabularx}
\end{table}
\clearpage
\workScheduleHeader
\tableObjective{4}{\oeFour}
\begin{table}[!ht]
  \begin{tabularx}{\textwidth}{|L|S|S|S|S|S|S|S|S|S|S|}
    \hline
    Realizar pruebas y validación con usuarios.           &  &  &  &  &  &  &  &  & \cc & \\
    \hline
    Realizar ajustes de acuerdo a validación de usuarios. &  &  &  &  &  &  &  &  & \cc & \\
    \hline
  \end{tabularx}
\end{table}
\tableObjective{5}{\oeFive}
\begin{table}[ht!]
  \begin{tabularx}{\textwidth}{|L|S|S|S|S|S|S|S|S|S|S|}
    \hline
    Realizar documentación técnica.    &  &  &  &  &  &  & \cc & \cc & \cc & \cc \\
    \hline
    Realizar documentación de usuario. &  &  &  &  &  &  &     &     & \cc & \cc \\
    \hline
  \end{tabularx}
\end{table}

\clearpage